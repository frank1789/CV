%%%%%%%%%%%%%%%%%%%%%%%%%%%%%%%%%%%%%%%%%
% Friggeri Resume/CV
% XeLaTeX Template
% Version 1.2 (3/5/15)
%
% This template has been downloaded from:
% http://www.LaTeXTemplates.com
%
% Original author:
% Adrien Friggeri (adrien@friggeri.net)
% https://github.com/afriggeri/CV
%
% License:
% CC BY-NC-SA 3.0 (http://creativecommons.org/licenses/by-nc-sa/3.0/)
%
% Important notes:
% This template needs to be compiled with XeLaTeX and the bibliography, if used,
% needs to be compiled with biber rather than bibtex.
%
%%%%%%%%%%%%%%%%%%%%%%%%%%%%%%%%%%%%%%%%%

\documentclass[a4,oneside]{friggeri-cv} % Add 'print' as an option into the square bracket to remove colors from this template for printing
\usepackage[utf8]{inputenc}
\usepackage[italian, english]{babel}
\usepackage{fontspec}
\usepackage{xcolor}
\definecolor{linkedin}{HTML}{1683BB}
\definecolor{skypeblue}{RGB}{0,175,240}
\definecolor{phonegreen}{HTML}{006B3C}
\usepackage{hyperref}
\hypersetup{
  colorlinks=false,
  pdfborder={0 0 0},
}

\newcommand{\LinkedinColour}{{\color{linkedin} \faLinkedin}}
\newcommand{\Email}{{\color{black} \faEnvelope \,}}
\newcommand{\Skype}{{\color{skypeblue} \faSkype}}
\newcommand{\Phone}{{\color{phonegreen} \faPhone}}

%note without number reference
\newcommand\blfootnote[1]{%
  \begingroup
  \renewcommand\thefootnote{}\footnote{#1}%
  \addtocounter{footnote}{-1}%
  \endgroup
}
\addbibresource{bibliography.bib} % Specify the bibliography file to include publications

\begin{document}
  % Your name and current job title/field
  \header{Francesco}{Argentieri}{junior Mechatronics Engineer}

%-------------------------------------------------------------------------------
%	SIDEBAR SECTION
%-------------------------------------------------------------------------------

\begin{aside} % In the aside, each new line forces a line break
  \section{contact}
  ~
  Circonvallazione Istoniense, 20
  Vasto (CH), 66054
  Italy
  ~
  \Phone \, +39 334 273 4061
  ~
  \Email \, \href{mailto:francesco.argentieri89@gmail.com}{francesco.argentieri89\\@gmail.com}
  ~
  \LinkedinColour \, \href{https://it.linkedin.com/in/francesco-argentieri}{Francesco Argentieri}
  \Skype \, \href{skype:my_username?add}{francesco\_argentieri}
  \section{languages}
  Italian--mother tongue
  English--upper intermediate
\end{aside}

%----------------------------------------------------------------------------------------
%	EDUCATION SECTION
%----------------------------------------------------------------------------------------

\section{education}
  \begin{entrylist}
  %------------------------------------------------
    \entrytitle{2015}{2020}
    {M. Sc. {\normalfont in Mechatronics Engineering}}
    {University of Trento}
    {Thesis "\emph{Enhancing UAV capabilities with machine learning on board}".}
    {Specialization in Mechanics--Mechatronics}
  %------------------------------------------------
    \entrytitle{2008}{2015}
    {Bachelor {\normalfont in Mechanics Engineering}}
    {Marche Polytechnic University}
    {Thesis "\emph{Structural analysis of an automotive hot formed sheet component with variable thickness}".}
    {Specialization in Energy-Thermomechanical}
  %------------------------------------------------
  \end{entrylist}

%----------------------------------------------------------------------------------------
%	WORK EXPERIENCE SECTION
%----------------------------------------------------------------------------------------
\section{experience}
  \begin{entrylist}
    %------------------------------------------------
    \entry{04/2019}{03/2020}
    {University of Trento}
    {Trento, Italy}
    {Thesis "\emph{Enhancing UAV capabilities with machine learning on board}".}
    {This project focuses on the activity of providing the drone's ability to
take advantage of the detection and classification of objects with TensorFlow
Lite. The whole system is run on ARM cortex-A53 and TPU processors for tensor
calculation, the project uses Raspberry Pi3b and Coral Dev-Board.}
    {\textbf{sofware:} Python, Tensorflow, Altair PBS (HPC), C++/Qt} 
    {\href{https://github.com/frank1789/MasterThesis}{\faGithub}}
    
    %------------------------------------------------

    \entry{09/2018}{11/2018}
    {University of Trento}
    {Trento, Italy}
    {"\emph{Rapid development CNN for image classification using fine-tuning
techniques and implementation on SoC systems}".}
    {Using framework like Keras is possible to develop refinement techniques
starting from already known models. Using architecture of a USB commercial
device, Intel Movidius neural compute stick, with low power consumption for
neural network execution on SoC systems such as Raspberry.}
    {\textbf{sofware:} Python, Keras, Tensorflow, Altair PBS (HPC)}
    {\href{https://github.com/frank1789/NeuralNetworks}{\faGithub}}
      
    %------------------------------------------------

    \entry{09/2017}{06/2018}
    {University of Trento}
    {Trento, Italy}
    {"\emph{Distributed robots mapping exploration}".}
    {Project for the final exam where we consider the problem of exploring an
environment unknown with a team of robots. As in the exploration of single
robots, the goal is to minimize the overall exploration time. The key problem to
solve in the context of multiple robots is that of choose the appropriate
destination points for the individual robots so that can explore different
regions of the environment simultaneously.}
    {\textbf{software:} Matlab, mex, C++, \LaTeX}
    {\href{https://github.com/frank1789/DistributedSystemProject}{\faGithub}}
      
    %------------------------------------------------
\end{entrylist}

\begin{entrylist}     
    \entry{05/2017}{08/2017}
    {University of Trento}
    {Trento, Italy}
    {"\emph{Helicopter's tail-boom and rotor vibration analysis}".}
    {This work performed during the master course of Modelling and Design with
Finite Elements, for the part about the course project. The purpose is to
present a consistent finite-element formulation, developed to predict the free
vibration characteristics of two different helicopters tail-boom structures.}
    {\textbf{software:} Ansys Mechanical (APDL), \LaTeX}
    {\href{https://github.com/frank1789/FEM-Analysis---Helicopter-s-Tail}{\faGithub}}
      
    %------------------------------------------------
      
    \entry{02/2015}{06/2015}
    {DIISM, Marche Polytechnic University}
    {Ancona, Italy}
    {Intership "\emph{ Structural analysis of an automotive hot formed sheet
component with variable thickness. }".}
    {In field of machine design developed a thesis during which I have developed
the ability to set and solve problems through the FEM simulations. The first
part was compare the component with variable thickness, verify its response to
static stresses respect a previous study where the same component had constant
thickness.  The second part of the work was characterized by research a method
to interface and study the molding's result simulated in Ls-Dyna. Results
obtained from the various simulations were compared, illustrating the advantages
and disadvantages encountered during development.}
    {\textbf{software: } Ansys Mechanical, Altair  HyperMesh, LsDyna, Qt, \LaTeX}
    {\href{https://github.com/frank1789/LsDynaToAPDL}{\faGithub}}
      
    %------------------------------------------------
    
  \end{entrylist}

%-------------------------------------------------------------------------------
%	SKILLS SECTION
%-------------------------------------------------------------------------------
\section{skills}
	\begin{entrylist} 
		\entryskills{Programming}{C++,  C, Qt, Python, Ruby, R, \LaTeX{}}
		
		\entryskills{Software}{Microsoft Office, Visual Studio Code}\\
		
		\entryskills{Package}{Matlab \& Simulink, Maple, Ansys, SolidWorks, HyperWorks}
		
		\entryskills{Other}{Internet networking, Arduino, Raspberry Pi}\\
		
		\entryskills{OS}{MacOS, Linux, Windows}
	\end{entrylist}

%-------------------------------------------------------------------------------
%	AWARDS SECTION
%-------------------------------------------------------------------------------

%\section{awards}
%
%\begin{entrylist}
%
%%------------------------------------------------
%
%\entry
%{2015--2015}
%{Postgraduate Scholarship}
%{School of Business, The University of California}
%{Awarded to the top student in their final year of a Bachelors degree. Mastered the art of filing accurate TPS reports.}
%
%------------------------------------------------
%\end{entrylist}

%-------------------------------------------------------------------------------
%	CERTIFICATION SECTION
%-------------------------------------------------------------------------------
\section{certification}
  \begin{entrylist}
	\entrycertificate{2018}{\href{https://bestr.it/award/show/657869a076ba7246b8667f9b2c26d9686d70e8be?ln=it}{Safety in the laboratory}}{University of Trento}
    \entrycertificate{2015}{Council of Europe Level B1 (PET)}{Cambridge English, University of Cambridge}
    
    %------------------------------------------------
    
  \end{entrylist}

%----------------------------------------------------------------------------------------
%	COMMUNICATION SKILLS SECTION
%----------------------------------------------------------------------------------------

%\section{communication skills}
%\begin{entrylist}
%
%------------------------------------------------
%
%
%\end{entrylist}

%-------------------------------------------------------------------------------
%	INTERESTS SECTION
%-------------------------------------------------------------------------------

%\section{interests}

%\textbf{professional:} automotive, design, robotic, R\&D, physics, space, aeronautics, IT   \textbf{personal:} cinema, cooking, sport, health and wellness, science and technology

\section{driver's license B}

%-------------------------------------------------------------------------------
%	PUBLICATIONS SECTION
%-------------------------------------------------------------------------------

%\section{publications}
%
%\printbibsection{article}{article in peer-reviewed journal} % Print all articles from the bibliography
%
%\printbibsection{book}{books} % Print all books from the bibliography
%
%\begin{refsection} % This is a custom heading for those references marked as "inproceedings" but not containing "keyword=france"
%\nocite{*}
%\printbibliography[sorting=chronological, type=inproceedings, title={international peer-reviewed conferences/proceedings}, notkeyword={france}, heading=bibheading]
%\end{refsection}
%
%\begin{refsection} % This is a custom heading for those references marked as "inproceedings" and containing "keyword=france"
%\nocite{*}
%%\printbibliography[sorting=chronological, type=inproceedings, title={local peer-reviewed conferences/proceedings}, keyword={france}, heading=bibheading]
%\end{refsection}
%
%%\printbibsection{misc}{other publications} % Print all miscellaneous entries from the bibliography
%
%%\printbibsection{report}{research reports} % Print all research reports from the bibliography
%
%%----------------------------------------------------------------------------------------
\blfootnote{“In compliance with the GDPR and Italian Legislative Decree no. 196
dated 30/06/2003, I hereby authorize the recipient of this document to use and
process my personal details for the purpose of recruiting and selecting staff
and I confirm to be informed of my rights in accordance to art. 7 of the above
mentioned Decree”.\\\\
\today}
\end{document}
