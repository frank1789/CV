%%%%%%%%%%%%%%%%%%%%%%%%%%%%%%%%%%%%%%%%%
% Friggeri Resume/CV
% XeLaTeX Template
% Version 1.2 (3/5/15)
%
% This template has been downloaded from:
% http://www.LaTeXTemplates.com
%
% Original author:
% Adrien Friggeri (adrien@friggeri.net)
% https://github.com/afriggeri/CV
%
% License:
% CC BY-NC-SA 3.0 (http://creativecommons.org/licenses/by-nc-sa/3.0/)
%
% Important notes:
% This template needs to be compiled with XeLaTeX and the bibliography, if used,
% needs to be compiled with biber rather than bibtex.
%
%%%%%%%%%%%%%%%%%%%%%%%%%%%%%%%%%%%%%%%%%

\documentclass[]{friggeri-cv} % Add 'print' as an option into the square bracket to remove colors from this template for printing
\usepackage[utf8]{inputenc}
\usepackage[italian, english]{babel}
\usepackage{fontspec}
\usepackage{xcolor}
\usepackage{fontawesome}
\newfontfamily\fatest{FontAwesome.otf} 
\definecolor{linkedin}{HTML}{1683BB}
\definecolor{skypeblue}{RGB}{0,175,240}
\definecolor{phonegreen}{HTML}{006B3C}
\usepackage{hyperref}
\hypersetup{
    colorlinks = false,
}

\newcommand{\LinkedinColour}{{\color{linkedin}{\fatest\char"F08C}}}
\newcommand{\Email}{{\color{black} \fatest\char"F003}} 
\newcommand{\Skype}{{\color{skypeblue} \fatest\char"F17E}}
%\newcommnad{\Github}{{\fatest\char"F09B}}

%note without number reference
\newcommand\blfootnote[1]{%
  \begingroup
  \renewcommand\thefootnote{}\footnote{#1}%
  \addtocounter{footnote}{-1}%
  \endgroup
}

\addbibresource{bibliography.bib} % Specify the bibliography file to include publications

\begin{document}

\header{Francesco}{Argentieri}{junior mechanic engineer} % Your name and current job title/field

%----------------------------------------------------------------------------------------
%	SIDEBAR SECTION
%----------------------------------------------------------------------------------------

\begin{aside} % In the aside, each new line forces a line break
\section{contact}
~
Circonvallazione Istoniense, 20
Vasto (CH), 66054
Italy
~
\color{phonegreen}\faPhone \, +39 334 273 4061
~
\Email \href{mailto:francesco.argentieri89@gmail.com}{francesco.\\argentieri89@gmail.com}
%\href{http://www.smith.com}{http://www.smith.com}
%\href{http://facebook.com/johnsmith}{fb://jsmith}
\LinkedinColour \href{https://it.linkedin.com/in/francesco-argentieri}{Francesco Argentieri}
\Skype \href{skype:my_username?add}{francesco\_argentieri}
\section{languages}
Italian--mother tongue
English--upper intermediate
%\section{programming}
%C++, Python, Qt
%Matlab \& Maple
%\LaTeX{}
\end{aside}

%----------------------------------------------------------------------------------------
%	EDUCATION SECTION
%----------------------------------------------------------------------------------------

\section{education}

\begin{entrylist}

%------------------------------------------------

\entry{2015--Now}
{M. Sc. {\normalfont Mechatronics Engineering}}
{University of Trento}
{\emph{Student}\\ Specialization in Mechanics--mechatronics}

%------------------------------------------------

\entry
{2008--2015}
{Bachelor {\normalfont Mechanics Engineering}}
{Marche Polytechnic University}
{Thesis "\emph{Structural analysis of an automotive hot formed sheet component with variable thickness}".\\
Specialization in Energy-thermomechanical}

%------------------------------------------------

\end{entrylist}

%----------------------------------------------------------------------------------------
%	WORK EXPERIENCE SECTION
%----------------------------------------------------------------------------------------

\section{experience}

\begin{entrylist}

%------------------------------------------------

\entry{9/2018–11/2018}
{ University of Trento} 
{Trento, Italy}
{\emph{Rapid development CNN for image classification using fine-tuning techniques and
 implementation on SoC systems}\\
Thanks to the use of framework like Keras is possible to develop refinement techniques starting
from already known models. There is discussion of the architecture of a USB commercial device,
Intel Movidius neural compute stick, with low power consumption for neural network execution
on SoC systems such as Raspberry. Finally, there are the problems and limitations that occurred
during the development and distribution of the software implemented.\\
\textbf{sofware:} Python 3.6, Keras, Tensorflow, Altair PBS (HPC)\quad
\href{https://github.com/frank1789/NeuralNetworks}{\faGithub}
}

%------------------------------------------------


\entry{9/2017–6/2018}
{University of Trento} 
{Trento, Italy}
{\emph{Distributed robots mapping exploration}\\
Project for the final exam where we consider the problem of exploring 
an environment unknown with a team of robots. As in the exploration of 
single robots, the goal is to minimize the overall exploration time. 
The key problem to solve in the context of multiple robots is that of choose 
the appropriate destination points for the individual robots so that can explore 
different regions of the environment simultaneously.\\
\textbf{software: } Matlab, mex, C++, \LaTeX \quad
\href{https://github.com/frank1789/DistributedSystemProject}{\faGithub}
}

%------------------------------------------------


\entry{5/2017–8/2017}
{University of Trento} 
{Trento, Italy}
{\emph{Helicopter's tail-boom and rotor vibration analysis}\\
This work performed during the master course of Modelling and Design with Finite
Elements, for the part about the course project.
The purpose is to present a consistent finite-element
formulation, developed to predict the free vibration characteristics of two
different helicopters tail-boom structures.\\
\textbf{software: } Ansys Mechanical (APDL), \LaTeX \quad
\href{https://github.com/frank1789/FEM-Analysis---Helicopter-s-Tail}{\faGithub}}

%------------------------------------------------

\entry{2/2015--6/2015}
{DIISM, Marche Polytechnic University}
{Ancona, Italy}
{\emph{Intership}\\
In the field of machine design developed a thesis during which it
has developed the ability to set and solve problems through the FEM simulations.\\
\textbf{software: } Ansys Mechanical, Altair  HyperMesh, LsDyna, Qt, \LaTeX \quad
\href{https://github.com/frank1789/LsDynaToAPDL}{\faGithub}
}

%------------------------------------------------

\end{entrylist}


%----------------------------------------------------------------------------------------
%	SKILLS SECTION
%----------------------------------------------------------------------------------------
\pagebreak
\section{skills}

\begin{tabular}{rlrl}
\textbf{OS} && \textbf{Package}\\
& MacOS, Linux, Windows  &&  Matlab \& Simulink, Maple, Ansys,\\ &&& SolidWorks, HyperWorks \\
\textbf{Software} && \textbf{Programming}\\
&Microsoft Office, iLife &&  C++,  C, Qt, Python, Ruby, R, \\ &&&\LaTeX{} \\
\textbf{Other}\\
&Internet networking, Arduino, Raspberry Pi\\
\end{tabular}\\

%---------------------------------------------

%\begin{entrylist}
%
%%------------------------------------------------
%
%%\entry
%%{2016}
%%{Project}
%%{M. Sc. Mechatronics Engineering, University of Trento}
%%{Excellent ability to work in team gained during the development of university projects.}
%%%------------------------------------------------
%
%\end{entrylist}

%----------------------------------------------------------------------------------------
%	AWARDS SECTION
%----------------------------------------------------------------------------------------

%\section{awards}
%
%\begin{entrylist}
%
%%------------------------------------------------
%
%\entry
%{2015--2015}
%{Postgraduate Scholarship}
%{School of Business, The University of California}
%{Awarded to the top student in their final year of a Bachelors degree. Mastered the art of filing accurate TPS reports.}
%
%%------------------------------------------------
%
%\end{entrylist}

%----------------------------------------------------------------------------------------
%	CERTIFICATION SECTION
%----------------------------------------------------------------------------------------

\section{certification}

\begin{entrylist}

%------------------------------------------------

\entry
{2015}
{Council of Europe Level B1 (PET)}
{Cambridge English, University of Cambridge}

%------------------------------------------------

\end{entrylist}

%----------------------------------------------------------------------------------------
%	COMMUNICATION SKILLS SECTION
%----------------------------------------------------------------------------------------

%\section{communication skills}
%
%\begin{entrylist}
%
%%------------------------------------------------
%
%\entry
%{2016}
%{Project: "\emph{High Mobility Robot}"}
%{University of Trento}
%{Excellent attitude for teamwork gained during the development of the same by applying the concurrent engineering.
%Also good ability to work with deadlines and long period stress.}
%
%%------------------------------------------------
%
%\entry
%{2015}
%{Oral Presentation}
%{DIISM, Marche Polytechnic University}
%{Presented the research I conducted for my Bachelor  mechanical engineering degree.}
%
%%------------------------------------------------
%
%\end{entrylist}

%----------------------------------------------------------------------------------------
%	INTERESTS SECTION
%----------------------------------------------------------------------------------------

%\section{interests}

%\textbf{professional:} automotive, design, robotic, R\&D, physics, space, aeronautics, IT   \textbf{personal:} cinema, cooking, sport, health and wellness, science and technology

\section{driver's license B}


%----------------------------------------------------------------------------------------
%	PUBLICATIONS SECTION
%----------------------------------------------------------------------------------------

%\section{publications}
%
%\printbibsection{article}{article in peer-reviewed journal} % Print all articles from the bibliography
%
%\printbibsection{book}{books} % Print all books from the bibliography
%
%\begin{refsection} % This is a custom heading for those references marked as "inproceedings" but not containing "keyword=france"
%\nocite{*}
%\printbibliography[sorting=chronological, type=inproceedings, title={international peer-reviewed conferences/proceedings}, notkeyword={france}, heading=bibheading]
%\end{refsection}
%
%\begin{refsection} % This is a custom heading for those references marked as "inproceedings" and containing "keyword=france"
%\nocite{*}
%%\printbibliography[sorting=chronological, type=inproceedings, title={local peer-reviewed conferences/proceedings}, keyword={france}, heading=bibheading]
%\end{refsection}
%
%%\printbibsection{misc}{other publications} % Print all miscellaneous entries from the bibliography
%
%%\printbibsection{report}{research reports} % Print all research reports from the bibliography
%
%%----------------------------------------------------------------------------------------
\blfootnote{“In compliance with the GDPR and Italian Legislative Decree no. 196 dated 30/06/2003, I hereby authorize the recipient of this document to use and process my personal details for the purpose of recruiting and selecting staff and I confirm to be informed of my rights in accordance to art. 7 of the above mentioned Decree”.}
\end{document}