%%%%%%%%%%%%%%%%%%%%%%%%%%%%%%%%%%%%%%%%%
% Friggeri Cover Letter
% XeLaTeX Template
% Version 1.1 (24/12/18)
%
% This template has been downloaded from:
% https://github.com/mlda065/friggeri-letter
%
% Original author:
% Matthew Davis, based on code by
% Adrien Friggeri (adrien@friggeri.net)
% https://github.com/afriggeri/CV
%
% License:
%    Copyright (C) 2012, Adrien Friggeri
%    Permission is hereby granted, free of charge, to any person obtaining a
%    copy of this software and associated documentation files (the "Software"),
%    to deal in the Software without restriction, including without limitation
%    the rights to use, copy, modify, merge, publish, distribute, sublicense,
%    and/or sell copies of the Software, and to permit persons to whom the
%    Software is furnished to do so, subject to the following conditions:
%    The above copyright notice and this permission notice shall be included in
%    all copies or substantial portions of the Software. THE SOFTWARE IS
%    PROVIDED "AS IS", WITHOUT WARRANTY OF ANY KIND, EXPRESS OR
%    IMPLIED, INCLUDING BUT NOT LIMITED TO THE WARRANTIES OF MERCHANTABILITY,
%    FITNESS FOR A PARTICULAR PURPOSE AND NONINFRINGEMENT. IN NO EVENT SHALL THE
%    AUTHORS OR COPYRIGHT HOLDERS BE LIABLE FOR ANY CLAIM, DAMAGES OR OTHER
%    LIABILITY, WHETHER IN AN ACTION OF CONTRACT, TORT OR OTHERWISE, ARISING
%    FROM, OUT OF OR IN CONNECTION WITH THE SOFTWARE OR THE USE OR OTHER
%    DEALINGS IN THE SOFTWARE.
%
% Important notes:
% This template needs to be compiled with XeLaTeX
% You may need to compile twice for the header to appear.
%
%%%%%%%%%%%%%%%%%%%%%%%%%%%%%%%%%%%%%%%%%

\documentclass[a4paper,english]{friggeri-letter}

\usepackage{babel}
\usepackage{hyperref}
\hypersetup{
    colorlinks=true,
    linkcolor=blue,
    urlcolor=blue,
}

\begin{document}
\header{Francesco}{Argentieri}{Mechatronics Engineer}
\address{
   Circonvallazione Istoniense, 20 \\
   Vasto (CH), 66054 \\
   Italy
}

\letter{
Alpine
}


\opening{Dear Hiring Manager}
I am Francesco Argentieri, express my interest in Simulation Development Engineer role advertised on your site. I got Mechatronic Engineering (M.Eng.) in March 2020 and qualified as an Industrial Engineer (24 July 2020 - I session - Italian Law)

Since June 2020, I have worked for Kineton as a software embedded developer.
My primary task is to develop Infotainment for a lightweight’s full-electric micro-car.
I gained experience in programming C++ and Qt regarding high-performance software. On the other hand, I also, am comfortable with interpreted languages such ad python for scripting and automation.

During my university career, I developed projects both in groups and individually.
I have acquired skills in design and modelling with Ansys, to conduct both static and dynamic simulations with FEM techniques. Furthermore,
I acquired a good knowledge of Matlab. Also, I got experience with symbolic manipulation software such as Maple.
In particular, the last one presents functionality close to commercial software Dymola.

I am looking for new challenges and the possibility of applying my knowledge to the world of motorsport.
I would invite you to view a short reel about the projects: \href{https://vimeo.com/414217599}{https://vimeo.com/414217599}, \href{https://vimeo.com/421445895}{https://vimeo.com/421445895} and my \href{https://github.com/frank1789}{GitHub page}.


\closing{
   Kind Regards\\
   Francesco Argentieri
}

\end{document}
